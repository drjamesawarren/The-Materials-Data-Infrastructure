\textit{Introduction}

In 2011, the United States Government announced the Materials Genome
Initiative (MGI), a multi-agency effort to accelerate the discovery, design,
development, and deployment of new materials for insertion in
manufaction products.    In order to achieve the goals laid out in the white paper
 \cite{mgiwhitepaper} accompanying the rollout of the MGI, a materials
innovation infrastructure must be constructed.  The infrastructure is
conceptually simple ***insert figure***, detailing the three main
components: computaional tools, experimental tools, and digital data.   

\subsection{What this work is not}


\subsection{A Use Case}

\subsection{The Bare Bones}
Data Infrastructure Needs for the Materials Genome Initiative:
\begin{enumerate}
\item Some sort of interface/api that the user interacts with to deposit their information such that (2) is enabled.  This entails
\begin{enumerate}
\item The establishment of repositories to store the data
\item The reposed data should then be marked up with sufficient metadata to inform someone else how the data was created, including attribution or provenance information relevant to citation 
(There is a likely role for standard metadata schemes for certain types of data in this requirement)
\item	Assignment of a persistent digital identifier (like the DOI for journals) so the data can be cited and discovered by others
\item Assignment of terms of use for the data (licenses), for both the
  creator of the data and the repository
\item Tools must be developed to enable 
\begin{enumerate}
\item the ingest of data from compuation or experiment for deposition
  in a repository
\item  Simplyfy the assignment of metadata either by
\begin{enumerate}
\item  well designed user interfaces or ideally 
\item automatically assigning some basic metadata through algorithmic
  analysis of the data set
\end{enumerate}
\item 	Assign persistent digital identifiers
\end{enumerate}
\end{enumerate}
\item  Some sort of interface/api that the user interacts with to find
  needed information.  This requires (at least)
\begin{enumerate}
\item	the registration of the availability of the data into some sort of “registry” to enable discovery without prior knowledge of the existence of the repository/specific data features.
\item Various types of policy enforcement  
\item	Terminologies and ontologies to enhance search 
\item	Tools to enable assessments of the relevance of the data to the question at hand, such as quick-look plotting or imaging capabilities
\end{enumerate}
\end{enumerate}




The technologies to do items 1+2 are not well developed in materials science. That is, if you were to do a Google search for X-ray data for a particular crystal, the results would be highly uneven at best, and largely worthless some of the time.  Let’s start by addressing (2).  We assume a researcher is developing a new material, and in order to do that, he needs a baseline of the existing materials data and models that will enable this design.  

